\section*{ABSTRACT}

\subsection*{Background}
This study presents two simulation modelling tools to support the organisation of networks of dialysis services during the COVID-19 pandemic. These tools were developed to support renal services in the South of England (the Wessex region caring for 650 patients), but are applicable elsewhere. 

\subsection*{Methods}
A discrete-event simulation was used to model a worst case spread of COVID-19 (80\%  infected over three months), to stress-test plans for dialysis provision throughout the COVID-19 outbreak. We investigated the ability of the system to manage the mix of COVID-19 positive and negative patients, and examined the likely effects on patients, outpatient workloads across all units, and inpatient workload at the centralised COVID-positive inpatient unit. A second Monte-Carlo vehicle routing model estimated the feasibility of patient transport plans and relaxing the current policy of single COVID-19 patient transport to allow up to four infected patients at a time.

%just dumped this here for the moment.
\subsection*{Results}

If current outpatient capacity is maintained there is sufficient capacity in the South of England to keep COVID-19 negative/recovered and positive patients in separate sessions, but rapid reallocation of patients may be needed (as sessions are cleared of negative/recovered patients to enable that session to be dedicated to positive patients). Outpatient COVID-19 cases will spillover to a secondary site while other sites will experience a reduction in workload. The primary site chosen to manage infected patients will experience a significant increase in outpatients and in-patients. At the peak of infection, it is predicted there will be up to 140 COVID-19 positive patients with 40 to 90 of these as inpatients, likely breaching current inpatient capacity (and possibly leading to a need for temporary movement of dialysis equipment).

Patient transport services will also come under considerable pressure. If patient transport operates on a policy of one positive patient at a time, and two-way transport is needed, a likely scenario estimates 80 ambulance drive time hours per day (not including fixed drop-off and ambulance cleaning times). Relaxing policies on individual patient transport to 2-4 patients per trip can save 40-60\% of drive time. In mixed urban/rural geographies steps may need to be taken to temporarily accommodate renal COVID-19 positive patients closer to treatment facilities.

%conclusions
\subsection*{Conclusions}
Discrete-event simulation simulation and Monte-Carlo vehicle routing model provides a useful method for stress-testing inpatient and outpatient clinical systems prior to peak COVID-19 workloads. 